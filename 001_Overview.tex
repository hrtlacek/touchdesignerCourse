%!TEX root = main.tex
%=================TD functions==================

\def\boldcommandlist{\@elt OP,\@elt OPs,}
\def\@elt#1,{%
 \expandafter\def\csname#1\endcsname{\textbf{#1}\xspace}
}
\boldcommandlist

\def\topColorList{\@elt TOP,\@elt TOPs,}
\def\@elt#1,{%
 \expandafter\def\csname#1\endcsname{\textcolor{TOP}{\textbf{#1}}\xspace}
}
\topColorList

\def\chopColorList{\@elt CHOP,\@elt CHOPs,}
\def\@elt#1,{%
 \expandafter\def\csname#1\endcsname{\textcolor{CHOP}{\textbf{#1}}\xspace}
}
\chopColorList

\def\sopColorList{\@elt SOP,\@elt SOPs,}
\def\@elt#1,{%
 \expandafter\def\csname#1\endcsname{\textcolor{SOP}{\textbf{#1}}\xspace}
}
\sopColorList

\def\datColorList{\@elt DAT,\@elt DATs,}
\def\@elt#1,{%
 \expandafter\def\csname#1\endcsname{\textcolor{DAT}{\textbf{#1}}\xspace}
}
\datColorList

\def\matColorList{\@elt MAT,\@elt MATs,}
\def\@elt#1,{%
 \expandafter\def\csname#1\endcsname{\textcolor{MAT}{\textbf{#1}}\xspace}
}
\matColorList


\def\compColorList{\@elt COMP,\@elt COMPs,}
\def\@elt#1,{%
 \expandafter\def\csname#1\endcsname{\textcolor{COMP}{\textbf{#1}}\xspace}
}
\compColorList


%===============================================


% \def\boldcommandlist{\@elt FS,\@elt OFS,\@elt RS,\@elt ORS,\@elt NR,\@elt NF,\@elt FNR,}
% \def\@elt#1,{%
%  \expandafter\def\csname#1\endcsname{\textbf{#1}\xspace}
% }
% \boldcommandlist
\color{textStd}

\addsec{Introduction/Notes}
\label{Notes}


\section{Notes} % (fold)
\label{notes}

Video playback

\subsection{Project Ideas}
\label{sub:ideen}
\subsubsection{Simple}
\begin{itemize}
	\item Hipster filter
	\item Audio Reactive Noise Ball
	\item Toon Shader
	\item Replicator Experiments
	\item Oscilloscope/Spectroscope
	\item leap motion
\end{itemize}

\subsubsection{Intermediate}
\begin{itemize}
	\item \glqq{}Auto-Cutter\grqq{}
	\item Particle stuff
	\item DMX
	\item big-data viz
\end{itemize}

\subsubsection{Long Term}
\begin{itemize}
	\item Mapping
	\item Music Video
	\item Aiko Aiko
	\item Procedural Modelling (Bicycle, House, ...)
	\item (3d) Multi-Touch Interface
	\item vj-app
	\item (Light) Show Simulation
	\item Color Grading Software
	\item video Analysis


\end{itemize}

\section{Course Structure} % (fold)

% \def\boldcommandlist{\@elt OP,\@elt OFS,\@elt RS,\@elt ORS,\@elt NR,\@elt NF,\@elt FNR,}
% \def\@elt#1,{%
%  \expandafter\def\csname#1\endcsname{\textbf{#1}\xspace}
% }
% \boldcommandlist


\subsection{Lecture 1 introduction and 2D-processing} % (fold)
\label{sub:lecture_1}
\begin{itemize}
	\item students show what they want to do in form of images in a toe file.
	\item showing off what TD can do, what I did so far.
	% \item history of TD (Houdini) and comparison to other similar software
	% \item possibly quick intro to latex/github for collaborative work on this
	\item Intro to TD
		\begin{itemize}
		\item getting help
		\item the GUI (quick overview)
		\item the \OPs
		\item in Depth: \TOPs and \CHOPs
		\item first guided projects
		% \item overview over useful \DATs
		% \item GUI more in depth (or rather do this along the way?)
		\item first python expressions (just something like \verb|absTime.frame/100|)
		\item a very quick look at GUIs
		\item output window handling and MovieFileOut \TOP. Possibly but rather not: off-line rendering.
		\end{itemize}
	% \item semester project groups and topics choice
\end{itemize}

\subsection{Lecture 2, Rendering Setup and Procedural Modeling}
\label{sub:lecture_2}
\begin{itemize}
	\item encapsulating Work, \COMPs
	\item converting between OP families, and why
	\item GUIs again
	\item Select \OPs
	% \item audio: filter \CHOPs, spectrum \CHOP, analysis \CHOP
	\item Render Setup
	\item some fun SOPs: copy, group, noise, facet, particle, magnet, force, twist, lattice
	\item feedback \TOP
	\item lights, light projection, shadows
	\item off-line rendering
\end{itemize}

\subsection{Lecture 3, Replication and Instancing, compositing and more project structure}
\label{sub:lecture_3}
\begin{itemize}
	\item instancing
	\item particle rendering
	\item replicator \COMP
	\item Clones
	\item render pass \TOP
\end{itemize}

\subsection{Lecture 4}
\label{sub:lecture_4}
\begin{itemize}
	\item Scripting
	\item Projection Mapping
	\item DATA/API visualization
\end{itemize}


\subsection{Lecture 5}
\label{sub:lecture_5}
% \itemize
Working on the final Project

\subsection{Lecture 6, project presentation}
\label{sub:lecture_6}


